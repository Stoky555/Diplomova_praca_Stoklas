V kontexte neustále sa rozvíjajúceho priemyslu a pokročilých technológií sme sa rozhodli venovať našu diplomovú prácu výzve rozpoznávania 3D identifikátorov v prostredí zmiešanej reality. Táto téma získava na dôležitosti s narastajúcim používaním rozšírenej a virtuálnej reality v priemyselných a komerčných aplikáciách. Súčasné technológie poskytujú pôsobivé možnosti pre interakciu s digitálnym svetom, avšak stále existujú výzvy, ktoré bránia ich širšiemu prijatiu a efektívnemu využitiu.

Naša práca sa sústreďuje na vývoj a testovanie nových metód rozpoznávania, ktoré by mohli zlepšiť interakciu s objektmi v rozšírenej realite tým, že umožnia presnejšiu a spoľahlivejšiu identifikáciu 3D objektov. Zameriavame sa na prototypovanie aplikácie pre rozpoznávanie a rozlišovanie 3D objektov a testovanie vytvorených 3D identifikátorov s využitím špičkových technológií a softvérových nástrojov, čo by mohlo viesť k výrazným zlepšeniam v priemyselných aplikáciách.

Prístup, ktorý sme zvolili, spočíva v detailnom štúdiu a porovnávaní existujúcich technológií a metodík rozpoznávania 3D objektov, návrhu a realizácii vlastnej metodiky a jej následnom testovaní v reálnom prostredí. Cieľom je nielen teoretický prínos do oblasti rozpoznávania objektov v zmiešanej realite, ale aj praktické aplikácie, ktoré môžu výrazne ovplyvniť spôsob, akým interagujeme s digitálnym svetom.

Táto práca predstavuje pokrok v aplikácii zmiešanej reality a prináša nové poznatky, ktoré by mohli byť využité v rôznych priemyselných odvetviach, od výroby po zdravotníctvo, čím sa otvárajú nové možnosti pre budúce výskumy a vývoj v tejto dynamicky sa rozvíjajúcej oblasti.