%EK: Zaver sa predpokladam bude cely prerabat
%V rámci nášho výskumu sme sa venovali rozvoju a implementácii metód rozpoznávania 3D identifikátorov v prostredí zmiešanej reality, pričom sme kládli dôraz na praktickú aplikovateľnosť týchto technológií v priemyselnom a logistickom sektore. Naše úsilie bolo zamerané na vývoj a prototypovanie nových technológií, ktoré by umožnili efektívnejšie identifikovanie a monitorovanie objektov v rôznych rovinách a prostrediach.

%Vytvorili sme a otestovali viacero prototypov 3D identifikátorov, ktoré boli navrhnuté tak, aby zvládali interakcie s objektami aj v náročných podmienkach. Výsledky našich testov potvrdili, že vytvorené riešenia sú spoľahlivé a poskytujú presné výsledky rozpoznávania, čo potvrdzuje ich potenciál pre priemyselné využitie. Taktiež sme prispeli k rozšíreniu teoretického poznania o aplikáciách zmiešanej reality, čo bolo podporené publikovaním výsledkov v prestížnych vedeckých časopisoch a získaním patentov.

%Naše výsledky taktiež naznačujú, že integrácia 3D identifikátorov do systémov zmiešanej reality môže značne zefektívniť procesy, ako je sledovanie a diagnostika zariadení v priemyselných podnikoch. Sme presvedčení, že naše práce otvárajú nové možnosti pre budúce výskumné a aplikované projekty v oblasti rozšírenej a zmiešanej reality, a prispievajú k inováciám v technologickom vzdelávaní a priemyselnej praxi.

%Uvedomujeme si, že napriek významnému pokroku existujú ďalšie výzvy, ktoré si vyžadujú ďalší výskum a rozvoj. Plánujeme preto pokračovať v práci na optimalizácii a rozširovaní funkcionalít 3D identifikátorov, aby sme mohli ešte viac prispieť k revolúcii v oblasti inteligentných výrobných a logistických systémov.