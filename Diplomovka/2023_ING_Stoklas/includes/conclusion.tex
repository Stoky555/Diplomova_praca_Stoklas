Táto diplomová práca poskytla hĺbkový pohľad na problematiku rozpoznávania 3D identifikátorov v prostredí zmiešanej reality. Cez analytický prístup a praktické experimenty bola vyvinutá a otestovaná metodika, ktorá značne rozširuje možnosti identifikácie objektov v priemyselných aplikáciách pomocou zmiešanej reality.

Výsledky experimentov potvrdzujú, že navrhované techniky poskytujú veľmi vysokú presnosť a efektivitu v rozpoznávaní 3D objektov, čo je kľúčové pre zlepšenie kontrolných a operačných procesov v priemyselnom prostredí. Tieto techniky prispievajú nielen k teoretickému porozumeniu, ale aj k praktickej aplikácii zmiešanej reality, čo predstavuje významný krok vpred pre integrované systémy v priemysle.

Na základe výsledkov práce je možné odporučiť ďalšie smery výskumu, najmä v oblasti rozvoja algoritmov umožňujúcich ešte rýchlejšiu a presnejšiu adaptáciu na zmeny v prostredí. Bolo by užitočné rozšíriť testovanie na ďalšie typy priemyselných aplikácií, kde by sa mohli skúmať potenciálne nové využitia vyvinutého softvéru. Taktiež by bolo užitočné rozšíriť množstvo modelov v databáze a otestovať úspešnosť rozpoznávania identifikátorov.

V neposlednom rade, diplomová práca potvrdila, že spojenie teoretického výskumu a praktickej aplikácie môže viesť k inovatívnym riešeniam, ktoré majú reálny dopad na priemyselné procesy a na spôsob, akým pristupujeme k problematike rozšírenej a zmiešanej reality. Ďalší vývoj a výskum v tejto oblasti by preto mal pokračovať s cieľom maximalizovať potenciál týchto technológií.