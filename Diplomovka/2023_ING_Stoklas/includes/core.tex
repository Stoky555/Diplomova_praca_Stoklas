\section{Teoretická časť diplomovej práce}

\subsection{Počítačom generovaná realita}
Počítačom generovaná realita alebo xR zahŕňa 3 typy realít, tými sú rozšírená realita (AR), virtuálna realita (VR) a zmiešaná realita (MR). Snaží sa o zmenu našej reality rozšírením, virtualizovaním alebo miešaním realít naskladaním alebo vnáraním počítačového textu alebo grafiky do našeho reálneho sveta a virtuálneho sveta alebo do oboch naraz. Všetky tieto 'reality' zdieľajú spoločné prekrývajúce sa rysy a požiadavky, každá má odlišné účely a základné technológie. XR hraje fundamentálnu rolu v metaverse - ďalšiu evolúciu internetu vďaka ktorej zlúčime reálny, digitálny a virtuálny svet do novej reality vďaka akejkoľvek dostupnej 'bráne' alebo teda prístroju ako je napríklad VR okuliare alebo AR inteligentné okuliare. Ako sme už spomínali, xR technológie zdieľajú fundamentálne rysy a to napríklad to, že základným kameňom všetkých nositeľných zariadení XR je schopnosť používať vizuálne vstupné metódy ako je sledovanie objektov, gest a pohľadu na navigáciu svetom a zobrazovanie kontextovo citlivých informácií. Hĺbkove vnímanie a mapovanie sú taktiež umožnené pomocou hĺbky a polohy. 

Taktiež môžeme extended reality nazývať aj pohlcujúcou technológiou.

https://www.arm.com/blogs/blueprint/xr-ar-vr-mr-difference

\subsubsection{Rozšírená realita - Augmented reality (AR)}

Rozšírená realita (AR) je integrácia digitálnych informácií s užívateľovým prostredím v reálnom čase. Rozšírená realita je používaná na buďto zmenu prirodzeného prostredia akýmkoľvek spôsobom alebo na poskytovanie dodatočných informácií používateľovi. Primárnym benefitom rozšírenej reality je to, že dokáže zmiešať digitálne a trojdimenzionálne komponenty s individuálnym vnímaním reálneho sveta. Rozšírená realita má mnoho využitia napríklad od pomáhania v rôznych situáciach a rozhodnutiach až po zábavu.

Rozšírená realita prináša vizuálne elementy, zvuky a iné senzorické informácie užívateľovi pomocou zariadenia ako je napríklad telefón alebo inteligentné okuliare. Tieto informácie su prekryté na zariadenie pre vytvorenie, aby mohli vytvoriť prepojený zážitok ako v reálnom svete avšak digitálna informácia ovplyvňuje a mení užívateľove vnímanie sveta. 

Zamestnanec Boeing Computer Services Research, Thomas Caudell, prvý krát použil názov rozšírená realita (augmented reality) v roku 1990 pri opise náhlavného prístroja, ktorý používali elektrikári pri skladaní komplikovaných káblových zväzkov. Jedným z prvých komerčných aplikácií technológie rozšírenej reality bola žltý prvá línia, ktorá sa začala používať počas televíznych futbalových podujatiach niekedy v roku 1998. 

Dnes sa rozšírená realita používa v Apple Smartwatches, čo sú inteligentné okuliare od firmy Apple, tie sú avšak stále vo fáze vývoja, v mobilných telefónových hrách a projekcie na čelných sklách automobilov sú najznámejšími konzumerskými produktmi rozšírenej reality. To však neznamená, že tam sa využitie rozšírenej reality končí. Táto technológia je používaná v mnohých odvetviach vrátane zdravotníctva, verejnej bezpečnosti, plynárenskom a ropnom priemysle, turizme a marketingu.

https://www.techtarget.com/whatis/definition/augmented-reality-AR

\subsubsection{VR}

Virtuálna realita je simulácia 3D prostredia ktorý umožňuje používateľovi objavovať a interagovť s virtuálnym prostredím spôsobom, ktorý sa čo najviac snaží napodobiť realitu, ako ju vnímame my ako užívatelia cez naše zmysly. Prostredie je vytvorené počítačovým hardvérom a softvérom a požívatelia musia byť vybavený zariadením ako je helma alebo okuliare na interakciu s virtuálnou realitou. Čím viac sa chce užívateľ ponoriť do takejto virtuálnej reality a blokovať všetko fyzické okolie, tým viac sú ochotní prijať "skutočnosť", že je to realita, aj keď má fantastickú povahu.

Priemysel okolo virtualnej reality je stále ďaleko od realizácie vízie vytvorenia dokonalého virtuálneho prostredia do ktorého keď zapojíme rôzne zmysly a  vnoríme sa do neho nebudeme vedieť rozoznať skutočnosť a fiktívny svet. Avšak, technológia má ešte veľmi veľa nedostatkov v poskytovaní takéhoto pocitu z virtuálnej reality, to však neznamená, že to nedokáže ani v budúcnosti. Virtuálna realita má odhad na obrovské využitie v rôznej škále priemyslov. 

Systémy virtuálnej reality sa líšia jeden od druhého signifikantným spôsobom, záležiac na ich spôsobe využitia a technológie ktoré používajú. Všeobecne však patria do nasledujúcich troch kategórií: 

Non-immersive (ne-immerzívne) sú typ zariadení virtualnej reality ktorá odkazuje typicky na 3D simulované prostredie ktoré je dostupne skrz počítačovú obrazovku. Prostredie môže taktiež generovať zvuk, to však záleží od programu. Užívateľ má nejakú kontrolu nad virtuálnym prostredím používaním klávesnice, myšky alebo iných zariadení, ale prostredie neinteraguje priamo s používateľom. Videohra je dobrým príkladom ne-imerzívneho typu virtuálnej reality, taktiež tu patrí napríklad aj webstránka , ktorá dovoľuje používateľovi nadizajnovať dekoráciu v izbe. 

Semi-immersive (čiasočne imerzívny) typ virtuálnej reality ponúka čiastočný zážitok z virtuálnej reality, ktorý je prístupný cez počítačovú obrazovku alebo nejaký typ okuliari alebo sluchatiek. Zameriava sa hlavne na vizuálne 3D aspekty virtuálnej reality a neinkorporuje fyzický pohyb v zmysle, v akom to dokáže len naozajstna realita. Bežným príkladom takejto čiastočne immerzívnej virtuálnej realiy je simulátor lietania, ktorý sa používa v leteckom a vojenskom priemysle na zaúčanie nových alebo trénovanie starších pilotov. 

Fully immersive (plne immerzívny) typ virtualnej reality ponúka najlepší zážitok z virtuálnej reality, kompletne ponorený používateľ je v simulovanom 3D prostredí. Inkorporuje zrak, zvuk a v niektorých prípadoch aj hmat. Sú tu dokonca aj experimenty s dodatočným zapojením čuchu. Používateľ má nasadené špeciálne zariadenie ako je helma, okuliare alebo rukavice, ktoré sú schopné plne interagovať s prostredím. Prostredie taktiež môže zahrňať využitie bežiaceho pásu alebo stacionárnych bicyklov na poskytnutie používateľovi zážitok z pohybu skrz 3D svet. Plne immerzívna technológia virtuálnej reality je oblasť, ktorá je stáre len v zárodku, ale už dokázala preraziť do herného priemyslu v nejakom slova zmysle aj do zdravotníckeho priemyslu a snaží sa zapájať do mnohá ďalších odvetví. 

https://www.techtarget.com/whatis/definition/virtual-reality

\subsubsection{MR}

Ekosystém zmiešanej reality je pomaly sa vynárajúca technológia ktorá začína byť na vzostupe a je záujem o jej vývoj čím ďalej tým viac. Je to technológia ktorá poskytuje fyzickú a digitálnu interakciu limitovanú iba našou predstavivosťou. Zmiešana realita je ďalšou vlnou v oblasti počítačov hneď po mainframoch, počítačoch a smartfónoch . Zmiešana realita sa pomaly stáva mainstreamom pre konzumerov a pre podniky. Oslobodzuje nás zo zážitkov viazaných na obrazovky poskytovaním inštikntívnej interakcie s dátami v našom životnom priestore a s našimi priateľmi. Online prieskumníci, počítajúc na stovky miliónov okolo celeho sveta, mali možnosť zažiť zmiešanú realitu pomocou ručne držaných prístrojov. Mobilnú rozšírenú realitu poskytuje najviac mainstream riešenie dodnes na sociálnych sieťach. Ľudia si to síce neuvedomujú, ale AR filtre, ktoré používajú napríklad na Instagrame sú v skutočnosti zážitok zo zmiešanej reality. Windows Mixed Reality posúva všetky tieto zážitky na vyššiu úrpveň s ohromujúcimi holografickými reprezentáciami ľudí, vysoká kvalita holografických 3D modelov a odhaľovanie sveta okolo.

Zmiešaná realita je zmesou fyzického z digitálneho sveta, odomknutím naturálnych a intuitívnych 3D ľudí, počítačov a enviromentálnej interakcie. Táto nová realita je založená na pokrokoch v počítačovom videní, v grafickom vyhodnocovaní, v displejových technológiach, v inputových systémoch a v cloudovych výpočtoch. Výraz zmiešaná realita bola prvý krát použitá v roku 1994 vo vedeckom článku od Paula Milgrama a Fumio Kishina s názvom: "A Taxonomy of Mixed reality Visual Displays". Ich článok sa snažil o objavovanie a vysvetľovanie konceptov virtuálneho kontinua a taxonomie vizuálnych displejov. Od vtedy prešla aplikácia zmiešanej reality mimo displeje pre poskytnutie:

Enviromentálne porozumenie: priestorové mapovanie a kotvy

Ľudské porozumenie: sledovanie rúk, sledovanie očí a hlasový vstup

Priestorový zvuk

Lokácia a pozíciovanie vo fyzickom aj vo vrituálnych priestoroch

Kolaborácia na 3D assetoch v priestoroch zmiešanej reality

V posledných dekádach sa vzťah medzi ľudmi a počítačmi postupne vyvíjal prostredníctvom vstupných metód, ktoré môžeme počítaču poskytnúť. Vznikla nová disciplína ktorá je známa ako interakcia človek-stroj alebo HCI (Human-computer interaction). Ľudský vstup môže teraz zahŕňať klávesnice, myšky, dotyk, atrament, zvuk a Kinect skeletové sledovanie. 
Kinect skeletové sledovanie je vývojový balíček s pokročilými senzormi, ktoré využívajú umelú inteligenciu. Poskytuje sofistikované počítačové videnie a hlasové modely. Zahŕňa v sebe senzor hĺbky, priestorové mikrofónové pole s videokamerou a senzor orientacie ako je všetko v jednom malom zariadení s viac režimami, možnosťami a sadami pre vývoj softvéru. 
Pokroky v oblasti senzorov a výpočtovej sile vytvárajú pre počítače nové typy vnímania prostredia založenom na pokročilých vstupných metódach. Toto je dôvod prečo API názvy, ktoré vo Windowse odhaľujú enviromentálne informácie sa bazývajú percepčné API.

https://learn.microsoft.com/en-us/windows/mixed-reality/discover/mixed-reality

\subsection{Industry 4.0}

Známa aj ako štvrtá priemyselná revolúcia, reprezentuje signifikantný rozdiel v tom, ako funguje priemysel a ako sa vytvárajú produkty. Je charakterizovaná integráciou digitálnych technológií, dátami a IoT (Internet of Things) v rámci tradičného manufaktúrneho procesu. Industry 4.0 nieje iba o automatizovaní rôznych úloh, ale vytváranie hlboko prepojeného a inteligentného systému, ktorý môže robiť rozhodnutia sám o sebe. Niektoré kľúčové technológie, ktoré poháňajú Industry 4.0 zahŕňajú umelú inteligenciu, strojové učenie, big data a pokročilá robotika.

https://www.linkedin.com/pulse/evolution-industry-what-40-history-industrial-sumit-rajan-vpaxf/

\subsubsection{História}

Prvá industriálna revolúcia začala na konci 18-teho storočia, s mechanizáciou textilného priemyslu a vynálezom parného stroja. V tomto období prebiehala tranzícia z agrárnej ekonomiky na priemyselnú ekonomiku, s parným strojom a mechanizáciou textilných fabrík hrála pivotnú rolu transformácia.

Druhá priemyselná revolúcia, ktorá sa začala v neskorších rokoch 19-teho storočia a skorších rokoch 20-teho storočia. Bola charakteristizovaná vývojom elektrickej energhie a výrobnými linkami. Masová produkcia a elektrifikácia priemyslu revolucionalizovala manufaktúry a vyrábanie produktov sa stalo viac dostupné a oveľa lacnejšie.

Tretia priemyselná revolúcia, často spomínaná ako digitálna revolúcia sa objavila v neskorších rokoch 20-tého storočia. Táto perióda bola poznačená rozšíreným používaním počítačov, internetu a automatizáciou. Tieto technológie vylepšovali komunikáciu a zefektívnili procesy, čo výrazne ovplyvnilo rôzne priemyselné podniky.

https://www.linkedin.com/pulse/evolution-industry-what-40-history-industrial-sumit-rajan-vpaxf/

\subsubsection{Súčasnosť}

Scéna založená na troch predchádzajúcich revolúciach, Industry 4.0 nás vedie do novej éry výroby. Buduje na digitalizácií a automatizácií tretej revolúcie, ale posúva to na novú úroveň. V tejto kapitole si povieme o kľúčových vlastnostiach Industry 4.0 a stručne ich popíšeme.

Konektivita - Najlepšie sa Industry 4.0 darí pod hlboko prepojených systémoch. Stroje a zariadenia komunikujú medzi sebou v reálnom čase pre neustály výmenu informácií a dátovej analýzy.

Dáta a analýza - Dáta sú základným kameňom Industry 4.0. Informácie sú zbierané z rôznych zdrojov a na analýzu sa využívajú pokročilé analytické pomôcky, ktoré dovoľujú výrobcom vytvárať inforomované rozhodnutia na optimalizáciu procesov.

Inteligentná výroba - Inteligentné fabriky sú vybavené strojmi, ktoré môžu robiť autonómne rozhodnutia, na základe predom poskytnutých inštrukcií. Inteligentné stroje sú vysoko adaptabilné a môžu sa prisposôbovať veľmi jednoducho na meniace sa požiadavky a podmienky.

Prispoôsobenie a efektivita - s pokročilými technológiami môžu výrobcovia produkovať vysoko prispôsobené produkty. Tie poskytujú dobré zázemie pre vysoko volatílny trh, ktorý je ovplyvňovaný dopytom po personalizovaných tovaroch.

Kľúčovými vlastnosťami Industry 4.0 je 7 kategórií ktoré riešia jednotlivé prvky, pre vylepšenie výroby. 

\begin{itemize}
       \item \textbf{Funkcionálne požiadavky}
       
       \begin{itemize}
         \item Cyber-fyzické systémy (CPS)
            \begin{itemize}
                \item Jadrom Industry 4.0 je integrácia fyzických strojov s inteligentnými digitálnymi systémami. Táto fúzia dovoľuje monitorovanie, analýzu a kontrolu nad priemyselnými procesmi.
            \end{itemize}
         \item Internet of Things (IoT)
            \begin{itemize}
                \item IoT zariadenia zbierajú dáta z výrobných strojov a senzorov, dovoľujúc im efektívne rozhodovanie sa a robenie rozhdonutí, predikovanie údržby a vylepšeným celkovým výkonom.
            \end{itemize}
         \item Big Data and analýza
            \begin{itemize}
                \item Masívny počet dát je generovaných a pokročilé analýzy sú zastúpene pre derivovanie dôležitých poznatkov z týchto dát. Tieto dátami poháňané robenie rozhodnutí optimalizuje produkciu a kvalitu.
            \end{itemize}
         \item Strojové učenie a Umelá inteligencia
            \begin{itemize}
                \item Tieto technológie sú používané na vývoj samo učiacich sa systémov, ktoré sa dokážu adaptovať a optimalizovať operácie postupom času.
            \end{itemize}
         \item Technológia digitálneho dvojčaťa
            \begin{itemize}
                \item Vytvára virtuálnu reprezentáciu fyzických assetov, povoľuje simulácie, testovanie a predikovanie údržby.
            \end{itemize}
         \item Cloud Computing
            \begin{itemize}
                \item Cloud dovoľuje jednoduche ukladanie dát, dostupnosť k nim a zdieľanie dát skrz celý výrobný proces a výrobný reťazec.
            \end{itemize}
         \item Inteligentné továrne
            \begin{itemize}
                \item Sú veľmi podrobne prepojené zariadenia kde stroje, produkty a systémy komunikujú a kooperujú medzi sebou.
            \end{itemize}
       \end{itemize}
    \end{itemize}

Industry 4.0 má veľký dopad na globálnej mierke. Vedie k viac efektívnemu a flexibílnemu výrobnému procesu. Redukuje prestoje (výpadky prevádzky), zvýšenie personalizácie produktov a znižuje náklady. Tento posun nielen, že transformoval výrobný sektor, ale taktiež vytvoril priestor pre nové pracovné príležitosti v technológiach a odbore dátovej analýzy.

https://www.linkedin.com/pulse/evolution-industry-what-40-history-industrial-sumit-rajan-vpaxf/

V článku Pandemic, War, Natural Calamities, and Sustainability: Industry 4.0 Technologies to Overcome Traditional and Contemporary Supply Chain Challenges popisuju ako kvôli mnohým problémom a úskaliam, ktoré museli podniky čeliť ako napríklad vojna, prírodné katastrofy, zvýšené množstvo smogu, pandémia museli podniky nájsť nové riešenia pri problémoch s logistikou a výrobou. Vďaka aplikácií Industry 4.0 technológií ako je umelá inteligencia, IoT, Big data a analýzu, blockchain, automatizáciu a robotizáciu liniek a mnohé iné vymoženosti vďaka štvrtej priemyselnej revolúcií sa zvýšila úspešnosť prežitia a zvýšila sa miera produktivity a tým aj ziskovosti podniku. Štúdia ukazuje, že aj keď je veľmi náročné predpovedať alebo detegovať úroveň narušenia zásobovania počas globálneho eventu, ale možnosť sa pripraviť a budovať dostatočne rezilientnú logistiku je v záujme každého podniku, ktorý sa chce presadiť na dnešnom trhu. Existuje obrovské množstvo problémom, ktoré môžu podniky čeliť a k obrovskému množstvu riešení sa vieme dopracovať práve vďaka štvrtej priemyselnej revolúcie Industry 4.0. 

https://www.mdpi.com/2305-6290/6/4/81 - logistics6040081

Industry 4.0 ako poslednou kapitolou (zatiaľ) v pokračujúcej ságe priemyselných revolúcií buduje na inováciach z minulosti. Prináša prepojenosť, dátami poháňajúce robenie rozhodnutí a inteligentné továrne do popredia. Integrácia technológií vo výrobnom sektore je dynamický proces a Industry 4.0 je testament neustále sa vyvíjajucého sa ekosystému priemyslu. 

https://www.linkedin.com/pulse/evolution-industry-what-40-history-industrial-sumit-rajan-vpaxf/

\section{Softvérové a hardvérové prostriedky}

\subsection{Herné enginy a modelovacie prostredia}

Herné enginy sú unikátnym prostriedkom pre vytváranie hier, animácií, aplikácií a tak ďalej. Je to vývojové prostredie s nástrojmi, ktoré optimalizáciami uľahčujú development skrz rôzne programovacie jazyky. Takéto enginy môžu obsahovať 2D a 3D grafický rendering, ktorý je kompatibilný s rôznym typom formátov. 

https://www.arm.com/glossary/gaming-engines

V súčasnej dobe je najpopulárnejšim herným enginom podľa hier, ktoré boli vydané na Steame, Itch.io, Unreal engin a Unity engine. Okrem týchto bližšie opíšeme aj aj Godot v ďalších kapitolách. Chceli by sme však spomenúť aj menej známe ako je napríklad CryEngine, ktorý mnoho pozná vďaka hernej fantázií menom Crysis a ďalších dielov a taktiež Frostbite enginy z dieľne Dice, ktorý sa preslávil sériou vojenských videohier z pohľadu prvej osoby Battlefield. V dnešnej dobe si mnoho hlavne herných firiem vyrába svoje vlastné enginy, aby nemuseli platiť časti zyskov zo svojich hier. Tieto čiastky sa môžu vyšplhať do výšky niekoľkých miliónov ak sa jedna o AAA hru. 

https://www.perforce.com/blog/vcs/most-popular-game-engines

Grafické modelovacie prostredie je softvér, ktorý nám umožňuje modelovanie 3D objektov alebo prípadne 2D obrázkov. My sa budeme bližšie zaoberať hlavne tými, ktoré poskytujú 3D možnosti modelovania. 3D modelovanie je teda počítačový grafický proces vytvárania matematickej reprezentácie trojdimenzionálnych obejektov alebo tvarov používaním špeciálneho softvéru. Digitálnou kópiou fyzického objektu nazývme 3D model a sú používané skrz rôzne premyselné odvetvia. 

Medzi najznámejšie patrí Blender, Autodesk Maya, Natron, Cinema 4D, Autodesk, CAD a ďalšie. Limitované možnosti poskytuje avšak aj Unity herný engine. 

https://www.autodesk.com/solutions/3d-modeling-software

\subsubsection{Unreal Engine}

Unreal engine je herným vývojovým prostredím. Bol vytvorený firmou Epic Games v roku 1988 na vývoj videohier z pohľadu prvej osoby. V dnešnej dobe je však využívaný na vytváranie rôznych iných typov žánrov ako je RPG, MMORPG a tak dálej. Používa programovací jazyk C++. V dnešnej dobe čím ďalej tým viac ľudí sa začína zaujímať o Unreal engine hlavne aj vďaka novým updatom, ktoré vydávajú pomerne často s veľmi dobrým ohlasom. 

Poskytuje používateľom rôzne mechanizmy, ktoré poskytujú platformu a dovoľujú im spustiť hru. Ako ďalšie, obsahuje rozsiahly systematický set nástrojov a editorov, ktoré pomáhajú používateľovi menežovať ich objekty v modifikovať ich na vytvorenie umeleckých diel do hry.

Unreal Engine poskytuje unikátnu grafickú kvalitu, často sa updatujúcim interface-om s novými nástojmi a možnosťami, jednoduchšiu prácu na programovanie prostredníctvom uzlového programovania názvom Blueprint. Tieto uzly pomáhajú používateľom na vytváranie akcií pre jednotlivé objekty. Takto poskytuje možnosť vytvoriť hru úplne bez písania skriptov a programovania a taktiež možnosť písanie skriptov a programovanie v jazyku C++.

https://www.educba.com/what-is-unreal-engine/

\subsubsection{Godot}

Popri mnohých herných enginoch na trhu, Godot je výborným herným enginom na vývoj low-cost hier a nieje tak populárny ako Unity alebo Unreal. Avšak, po fiasku Unity s novou stratégiou monetizovania sa mnoho developerov začalo obzerať po aleternatíve a Godot sa stal ich novým programovacím prostredím. Prvý krát sa objavil v roku 2014 ako cross-platform herný engine orientovaný na 2D a 3D vývoj hier. Pre naše účely však neposkytuje takú škálu integrovaných AR knižníc ako by sme potrebovali.

https://gamedevacademy.org/what-is-godot/

\subsubsection{Unity}

Unity je herný engine pre takmer každého, kedže je pomerne jednoduché sa v ňom naučiť pracovať a vyvíjať aplikácie, na to stačí základná orientácia v tomto engine, pochopenie štruktúr klás pri implmentácií rôznych doplnkov a funkcií a schopný počítač na spustenie kompilácie a hry a chuť vytvárať nové veci alebo vylepšovanie už starých prostredníctvom kontribúcie do projektov napríklad na githube.

Mnoho ľudí, ktorý sa aspoň trochu zaujímajú o hry túžilo alebo stále túži o svojej vlastnej hre. V súčasnosti sa bariéry dvíhajú a čas, investície a výkon sú veľmi žádané, aj kvôli mobilným hrám. Jedným z týchto enginov je Unity engine. Nie je len pre malé vývojové štúdia (alebo jednotlivcov), medzi mnohé herné tituly vytvorenými v Unity hernom engine patrí napríklad Escape from Tarkov, Pokemon Go, Call of Duty Mobile, Cuphead, Cities: Skylines.

Unity poskytuje široké množstvo nástrojov na vývoj mobilných hier a iných typov softvéru. Dokážeeme implementovať aplikácie na škálu rôznych platforiem ako je windows, Android, iOS, Linux a macOS. 

Unity je založený na programovacom jazyku C\# a podporuje vytváranie 2D, 3D a iných typov softvérov. Štatistiky z roku 2022 ukazujú, že 70\% z top 1000 mobilných hier je vytvorených v Unity hernom engine. K dnešnému dňu vygenerovalo Unity viac ako 1.1 miliardy dolárov prostredníctvom reklám v aplikáciach. 

Engine funguje skrz assety, ktoré môžeme importovať alebo stiahnuť z Unity marketu, nachádza sa tam mnoho platených ale nájdu sa tam aj tie zadarmo. Unity taktiež poskytuje vývoj sofvéru rozšírenej reality vďaka knižniciam kompatibilných s Unity, ktoré sme opisovali v kapitolách vyššie. Taktiež unity poskytuje obrovské množstvo tutoriálov priamo na ich stránke a taktiež je veľké množstvo materiálov o unity prístupné zadarmo na internete, prípadne za platobnými bránami rôzne prémiové kurzy.

Pre prístup do unity je možné si vybrať z rôznych typov subskripčných plánov. Medzi prvými je personálny, tento plán je zadarmo a poskytuje prístup do vývojovej platformy, do Unity Visual Scriptingu, Unity Version Control až do troch používateľov a rôzne ďalšie. Tento plán je sčasta dostačujúci pre každého začínajúceho aj pre menšie herné štúdia. Plus stojí 399 dolárov za rok a Pro stojí približne 1800 dolárov na rok obsahujú ďalšie nástroje ako je nasadenie aplikácií na konzoly, vývoj AR, support servis pre zákazníkov a tak ďalej. Enterprise je prispôsobená kvóta, ktorú určuje unity na základe zárobkov hry. Posledne taktiež existuje Industry plán za okolo 4500 dolárov, ktorý je primárne pre priemyselné podniky. Je vhodné podotknúť že vzhľadom na nedávne udalosti sa Unity potýkalo s veľkou odozvou po vyhlásení novej monetizačnej stratégie, ktorá vrhla na Unity mnoho negativity.  

https://www.androidpolice.com/what-is-unity/

\subsubsection{Blender}

Blender 3D je open-source softvérová aplikácia, ktorá prináša silné a versatilné nástroje na vytváranie 3D modelov. Ponúka široké spektrum doplnkov a funkcií pre rôzne aspekty tvorby 3D, vrátanie modelovania, rigovania, animácií, simulácií, renderingun, kompozície a sledovanie pohybu. Je vyvíjaný firmou Blender Foundation a je dedikovaná komunite. Je to veľmi populárny nástroj, ktorý je známy ako medzi začiatočníkmi, tak aj medzi profesionálmi v 3D kontentovom priemysle. 

Blender bol spočiatku vydaný firmou Blender Foundation v roku 2002 ako zadarmo dostupné open-source softvérové riešenie. Jeho misia bola priniest na svet najlepšiu 3D počítačovú grafickú technológiu na svete dostupnú pre umelcov a tvorcov. Po čase sa Blender vyvíjal a rástol, s priebežnými updatmi a vylepšeniami vďaka kolaboratívnemu úsiliu komunity. 

Práca v nástroji Blender je pomerne jednoduchá na naučenie a obsahuje obrovské množstvo skratiek, ktoré uľahčujú tvorbu 3D modelov, avšak interface je veľmi komplexný a obsahuje veľké množstvo funkcií, že sa zo začiatku môže užívateľ cítiť preťažený. Obsahuje v sebe množstvo nástrojov pri budovaní 3D a nie sú potrebné dodatočné nástroje ako pri iných 3D modelovacích nástrojoch. 

Jednou z nevýhod je limitovaná integrácia s nástrojmi z tretej ruky, teda si nerozumie veľmi dobre s externými nástrojmi a pluginmy, ako iné komerčné značky. To sa však snaží kompenzovať budovaním vlastných knižníc a addonov, ktoré rozširujú jeho funkcionalitu.

https://www.crealitycloud.com/blog/reviews/what-is-blender

\subsection{Knižnice pre AR / MR}
Keďže sa rozšírená realita stáva čím ďalej, tým viac populárnejšou znamená to, že je potrebné vyrábať viac aplikácií, ktoré s ňou dokážu pracovať. Túto dieru na trhu začali vypĺňať rôzny technologický giganti ako je napríklad aj Google (ARCore) a Apple (ARKit) vlastnými knižicami na vytváranie takýchto aplikácií. Na trhu je však aj mnoho iných knižníc, ktoré im dokážu konkurovať. Tými sú napríklad Vuforia, Wikitude a Visionlib. Tieto si predstavíme jednotlivo v ďalších sekciách.

\subsubsection{ARCore}

Arcore je Google platforma slúžiaca na vývoj aplikácií, ktoré podporujú ambientálny zážitok s rozšírenou realitou. Používa rôzne API (Application Programming Interface). 

https://developers.google.com/ar/develop

Čo je API.

API je súbor pravidiel, protokolov a knižníc, ktoré umožňujú rôznym sofvérovým programom komunikovat a integrovať medzi sebou. Jedná sa o rozhranie, ktoré spokythuje určité funkcie a možnosti pre iné programy, aby s nimi mohli pracovať bez toho, aby museli poznať vnútornú štruktúru alebo detaily týchto funkcií. Využíva sa hlavne pri webových aplikáciach. Pričom web slúži ako prostredník pre prácu s danou aplikáciou a API je implementované v pozadí ako Backend danej aplikácie. API ďalej pracuje s databázou, vykonáva rôzne výpočty, úpravu dát, implemtnáciu logiky atď.

https://www.rascasone.com/cs/blog/co-je-api

ARCore povolí telefónu vnímanie prostredia a pochopenie sveta aby mohol následne interagovať s informáciami. Niektoré API sú dostupné skrz Android a iOS pre získanie AR skúseností. ARCore používa 3 kľúčové prvky na integrovanie virtuálneho kontentu s reálnym svetom, ktoré vníma skrz kameru. Tieto prvky sú:
- Motion tracking alebo sledovnie pohybu, to mu dovoľuje pochopiť a sledovať jeho pozíciu relatívne naprieč svetom. 
- Enviromental understanding alebo pochopenie okolia dovolí mobilnému telefónu detegovať veľkosť a polohu všetkých typov povrchov - horizontálne, vertikálne a sklonené povrchy ako je napríklad zem, počítačový stôl alebo radiátor na stene či chladničku v kuchyni.
- Light estimation alebo osvetlenie umožňuje telefónu odladnúť aktuálne svetelné podmienky prostredia.

ARCore je aktuálne navrhnutý pre prácu skrz rôznymi zariadeniami, ktoré majú integrovanú aspoň Anrdoid verziu 7.0 (Nougat) a novšie verzie. 

https://developers.google.com/ar/develop

Ako ARCore funguje?

ARCore nepotrebuje žiadne špeciálne senzory, keďže je to knižnica, ktorá je zameraná hlavne na mobilné telefóny, ktoré sú bežne dostupné pre každodennych konzumerov s bežnými kamerami. Inak povedané niesu potrebné žiadne vysoko úrovňové kamery používané na vedecké účely. (Existuju knižnice ako je Project Tango, na túto knižnicu sú potrebné vysokoprofilové kamery a špeciálne senzory). To teda znamená, že sa spolieha len na kameru na mobilnom telefóne a senzor pohybu ako je napríklad gyroskop, akcelerometer a komplikované softvérové triky implementované v tejto knižnici. ARCore následuje sériu fundamentálnych konceptov aby dokázala táto knižnica úspešne interpetovať čo kamera vidí a na základe týchto informácií poskytuje užívateľovy integrovaný zážitok s rozšírenej reality. 

https://www.xda-developers.com/arcore/

ARCore knižnica využíva  simultánne lokalizovanie a mapovanie (simultaneous localization and mapping), alebo inak povedané SLAM, na chápanie kde presne sa mobilný telefón nachádza vzhľadom na svet okolo neho. Dokáže vypočítať zmeny v polohe detegovaním vizuálne rozdielných prvkov zachytených na kamerovom zázname a potom to použiť ako zachytávacie body, podľa toho vie či sa zmenila poloha a jednotlivé charakteristické znaky danej lokácie. Využíva teda tieto charakteristické znaky na zachytávanie plôch alebo horizontálnych, prípadne vertikálnych povrchov a používa ich pre vytvorenie dodatočného kontextu. 

Pre odvodenie polohy kamery (poloha a orientácia) vzhľadom na prostredie v čase sa páruje dodatočne s inertiálnymi meraniami z IMU zariadenia (Inertial Measurement Unit). 

https://www.xda-developers.com/arcore/

IMU je zariadenie alebo senzor, ktorý meria zrýchlenie a uhly rotácie (gyroskopické hodnoty) v troch rôznych osiach. 

https://www.vectornav.com/resources/inertial-navigation-articles/what-is-an-inertial-measurement-unit-imu

Používaním týchto informácií a kontextu môže vývojár rendrovať veci na kamerovom zázname a spraviť to tak, že to vyzerá akokeby to bolo súčasťou reálneho sveta. Taktiež dokáže ovplyvniť koľko svetla sa na ploche nachádza a používaním tohoto kontextu môže rendrovať obrázok, ktorý vyzerá svetlejšie alebo tmavšie na základe toho koľko svetla sa zachytáva v kamere. Ako výsledok je teda realistickejšie zobrazenie umelo pridaného objektu.

Tieto fundamentálne koncepty sú len povrchovo zhrnuté, v pozadí sa odohrávajú veľmi komplikované výpočty a často využívajú rôzne fyzikálne zákony, ktoré následne program prepočítava do výsledku, ktorý môžeme vidieť na zázname z kamery.

Jednou zo zaujímavých doplnkov Google ARCore je taktiež Cloud Anchors, to dovoľuje položiť, ktorý je viditeľný v rámci Cloudovej relácií a aj iný používatelia, ktorý sa napoja na túto reláciu môžu vidieť tento objekt na tom istom mieste.

https://www.xda-developers.com/arcore/

\subsubsection{ARKit}

ARKit je knižnica vytvorená firmou Apple a využíva iOS zariadenia. Dovoľuje, rovnako ako ARCore, developerom produkovať aplikácie, ktoré umožňujú interakciu so sveom využívaním kamery a senzorov. 

Apple videl veľké využitie AR v budúcnosti mobilných telefónov, preto sa rozhodli vyplniť dieru na trhu a vytvoriť si vlastnú knižnicu, vďaka ktorej by mohli poskytnúť developerom vhodnú knižnicu na vytváranie AR aplikácií. Túto knižnicu predstavili v oku 2017 ako súčasť iOS 11. Taktiež zároveň oznámili dalšiu verziu v iOS 12 a v podstate vydávajú novu verziu ARKit vždy keď vyjde nova verzia iOS. 

Má veľmi veľa podobných prkov ako ARCore, preto nemusíme znovu vysvetľovať jednotlivé koncepty ako AR funguje. Pozrieme sa však na podstatné rozdiely oproti ARCore.

ARKit využíva technológiu zvanú Visual Inertial Odometry (VIO) na sledovanie sveta okolo iPadu alebo iPhonu, to im dovoľue vnímať ako sa pohybuje mobilný telefón priestorom. ARKit využíva tieto dáta nielen na analýzu rozpoloženia izby alebo priestoru, ale taktiež deteguje horizontálne plochy ako je stôl alebo podlaha. Následne vďaka tomu môžeme položiť objekt na tieto plochy rovnako ako pri ARCore, ak to je vec ktorú chceme dosiahnť využívanim takejto aplikácie.

Predpoklady pre vývoj tejto knižnice sú, že v budúcnosti bude môcť človek využívať rozšírenú realitu aj v rámci toho, že bude mať okuliare, predstavme si teda Apple AR okuliare, ktoré následne môžu rozširovať našu realitu nosením takéhoto zariadenia, nie len teda snímaním kamery a využívaním konkrétnej aplikácie.

ARKit rovnako ako Google ARCore (v rámci Cloud Anchors) má možnosť zdieľať objekty položené na povrchu a ak sa druhé zariadenie pridá do takejto relácie môže vidieť rovnakú AR scénu. Taktiež môžu pozastaviť a znovu spustiť túto reláciu aby umožňovali lepší zážitok používateľom vrátiť sa do virtuálneho scenária. Ako príklad môžeme uviesť predizajnovanie obývačky. Užívatelia môžu položiť napríklad do rohu obývačky skriňu, ktorú uvidia aj iné zariadenia pripojené do tejto relácie. Neskôr ak sa k tomu znovu vrátia napríklad o deň a nasnímaju povrchy znovu uvidia túto skriňu na rovnakej pozícií a prípadne ju premiestniť alebo pridať iné prvky do obývačky v rámci niekoľkých dní.

https://tinyurl.com/yym6rucb

\subsubsection{Vuforia}

Vuforia je multi-platformová knižnica pre Zmiešanú realitu (MR) a rozšírenú realitu (AR) s plným trackovaním a implemtnáciou na rôznych typoch hardvéroch vrátane mobilných zariadení a Head Mounted Displays (HMD) (to sú v podstate okuliare využívajúce zmiešanú alebo rozšírenú realitu) ako je Microsoft HoloLens.

Vuforia má samostatný softvér na vývoj aplikácií alebo je k dispozícií aj ako knižnica integrovaná s Unity vývojovým enginom. TO dovoľuje konštrukciu vizuálnych aplikácií a hier pre Android a iOS utilizovaním drag-and-drop (ťahaj a spusť) toku práce (workflowu).

Medzi hlavné výhody Vuforie je Android, iOS, UWP a Unity Editor podpora Vuforie. Taktiež spomínaný PTC development toolkit, ktorý dovoľuje monitorovanie špecifických fotiek, modelov, objektov alebo 3D skenov.

Aplikácie Vuforie môžeme spúšťať na iOS aj Android zariadeniach a dokonca aj na starších iPhone modeloch, ktoré nepodporujú ARKit. Ak to dané zariadenie dovoľuje, Vuforia využíva ARKit alebo ARCore, ak nedovoľuje využíva svoju platformu.

Jadrom vývoja pomocou Vuforie je hlavne sledovanie obrazu a objektov. Vuforia podporuje nahrávanie modelov, obrázkov, skenov objektov a iných typov cieľov pre detekciu. Týmto môžeme nahrávať detekciu objektov zo špecifikovaných cieľov pre rozpoznávanie.

Tento nástroj pomáha pri identifikácií s rôznymi kategóriami ako su krabice, cylindre a plochy. Taktiež rozpoznáva text a prostredie. VuMark je hybridný typ fotky a QR kód, je to iný nápomocný doplnok, ktorý Vuforia poskytuje. Vuforia Object Scanner umožňuje vývojárom skenovať a vytvárať cieľové objekty. 

Krátke vysvetlenie jednotlivých doplnkov Vuforie:

- Model Targets - Modelové ciele - používa 3D model na rozpoznávanie predmetov, ktorým základom je ich samotná forma. Môžeme nainštalovať AR kontent na viacerých objektoch z rôznych uhlov na veľkom rozsahu produktov ako je industriálna výroba, autá a hračky.

- Image Targets - Obrazové ciele - obrázky sú najjednoduchším spôsobom ako pozmeniť reálne materiály na rovné povrchy ako sú napríklad stránky magazínu, iné obrazy alebo zberateľské kartičky.

- Multi Targets - Multiciele - používa sa na objekty s viacerými hranami a rovnými povrchmi alebo zahŕňajú viaceré obrázky. Multiciele zahŕňajú obaly produktov, plagáty a maľby.

- Cylinder Targets - Cylinderové ciele -  Môžeme použiť cylindrové ciele pre umiestnenie AR informácie na cylindrické alebo kužeľové objekty. Sú ideálne na plechovky od sódy, flaše a tuby s vytlačenou grafikou.

- Object Targets - Objektové ciele - Skenovanie objektu produkuje objektové ciele. Sú excelentným výberom pre hračky a iné predety s konzistentnou formou a bohatými povrchovými detailami. 

- VuMarks - Môžeme použiť VuMarks na identifikáciu a pridávanie kontentu na skupinku objektov. Je to excelentná metóda na doplnenie produktových liniek, inventáre a stroje s informáciami a kontentom.

https://www.codingninjas.com/studio/library/introduction-to-vuforia-sdk

\subsubsection{Wikitude}

Wikitude SDK pre Unity je plugin pre Unity3D, ktorý pridáva funkcionalitu rozšírenej reality do Unity prostredia. Je to optimalizované na detekciu a trackovanie obrazov a objektov, ktoré sa používajú v zážitkoch rozšírenej reality, buď spolu s natívnym AR frameworkom ako je ARCore, ARKit, HoloLens, wrapermi ako AR Foundation alebo ako samostatný nástroj na poskytovanie AR zážitkov.

Používaním Wikitude SDK pre Unity môžeme využívať tieto vlastnosti knižnice:

Sledovanie obrazov na zariadení

Nasledovnými príkladmi môžeme ukázať rozpoznávanie 1

Sledovanie cylindrických objektov

Sledovanie obrazov z cloudového servra

Sledovanie viacerých obrazov súčasne

Sledovanie objektov a scén

Sledovanie viacerých objektov súčasne

Kombinovanie viacerých trackerov z vyššie uvedených v jednej relácií AR

Môžeme využívať všetky uvedené naraz v kombinácií s pozičným sledovaním pomocou ARCore, ARKit a AR Foundation

Wikitude SDK knižnica pre unity je rozšírením funkcionalít ARKit knižnice a ARCore knižnice doplnkami, ktoré niesu prítomné v natívnych verziách týchto AR frameworkov alebo prichádza s inými kvalitativnými štandardami v porovnaní s implementáciou vo Wikitude SDK. V tomto smere, Wikitude SDK môžeme vidieť ako rozšírením ARkit knižnie a ARCore knižnice. Používanie týchto (ARCore knižnice a ARKit knižnice) nie je povinne pre používanie Wikitude SDK.

Wikitude SDK môže využívať AR Foundation od Unity alebo fungovať úplne samostatne. Kedže AR Foundation je skvelým wrapperom ARCore knižnice a ARKit knižnice a iných AR frameworkov, neprináša žiadne funkcionality navyše.

https://www.wikitude.com/external/doc/expertedition/\#key-features

\subsubsection{AR Foundation} 

AR Foundation je cross-platformová API pre development aplikácií vyvíjaných v Unity. Pomocou tohoto API dovoľuje vývojárom používať softvér a vydávať príkazy pre rozhrania (interface). Sú široko používané pre stolné počítače, notebooky alebo mobilné zariadenia ako je iOS a Android.

AR Foundation umožňuje vytváranie zážitku rozšírenej reality pre mobilné telefóny, tablety, nasaditeľné AR zaradenia cez pluginy, ktoré poskytuje Unity. Pluginy vrátane ARCore knižnicu, ARKit knižnicu, Magic Leap a HoloLens. Každý takýto plugin je kompatibilný s rozličnými operačnými systémami ako je iOS, Android a produkty od Microsoftu.

Používaním Unity, developeri môžu vykonať build na ich vlastných modeloch alebo importovaných prefaboch z Asset store v Unity na vytváranie projektu. Používaním skriptov v c\# môže používateľ dizajnovať rozličné cesty ako interagovať a používať nástroje ako je sledovanie zariadenia, sledovanie plôch a sledovanie tváre. Skrz AR Foundation, používateľ môže jednoducho získať prístup do rozličných doplnkov, ktoré developerovi umožnia exportovať ich projekty na rozličné hardvérove typy súčasne. Tento nástroj dovoľuje vývojárom dizajnovať ich AR zážitok jednoducho s rozličnými pluginmi a Unity skúsenosťami.

https://docubase.mit.edu/tools/ar-foundation/

\subsubsection{VisionLib}

VisionLib dovoľuje vývojárom vytvárať industriálne aplikácie využivajúce rozšírenú realitu na úrovni priemyselných technológií trackovania počítačového videnia. Táto knižnica využiva takzvaný Enhanced Model Tracking. VisionLib patrí medzi najviac uznávané AR trackingové knižnice pre industriálne a priemyselné využitie. Počítačové videnie a trackovanie modelov sú kľúčové v každej skúsenosti s rozšírenou realitou spojenou s fyzickými objektami alebo teda hľadanými objektmi, tie sú rozšírené a doplnené digitálnymi informaciami. Či už na opravu a údržbu, AR založenými tréningovými kurzami, marketingové alebo predajné ciele, nič z týchto by nefungovalo, bez precízneho a spoľahlivého objektového detegovania a sledovania.

VisionLib je teda multi-platforomová AR trackovacia knižnica, ktorá integruje celú škálu algoritmov potrebných na sledovanie cieľov v rozšírenej realite.  

S Visionlib môžete vytvoriť aplikácie v Unity3D a nasadiť ich na Windows(32-bit aj 64-bit), Android (arm7, Android 4.4+) a iOS zariadenia. Taktiež môžete integrovať aplikáciu aj na iné operačné systémy používaním kódu v čistom programovacom jazyku C. Dodatočne poskytuje simplifikovaný objective-C interface, ktorý dovoľuje používanie SDK na programovanie natívnych macOS a iOS aplikácií. 

Jadro SDK sa sklada z dynamických a statických knižníc, ktoré môžeme pripnúť k aplikácií. Aktualne API ponúka čistú a zostrihaný prístup do funkcionality jadra enginu. Je to teda multi-platformová, na širokú škálu zariadení používateľná knižnica podporujúca vývoj v Unity3D a taktiež na týchto natívnych plaformách ako je iOS, Android, UWP, macOS a Windows. Čo sa týka zariadení podporuje desktopové zariadenia, tablety, smartfóny a smart okuliare. 

Dizajnovo sú trackery v knižnici Visionlib a dáta používané na sledovanie práce bez prerušenia a to lokálne bez potreby pripojeniana na internet alebo na pripojenie k VisionLib on-line služieb.

Visionlib môžeme orchestrovať a kontrolovať jednotlivé základné sledovacie správania cez konfiguračný súbor sledovania, označenie týchto súborov je koncovkou vl, napríklad TrackingConfigurationFile.vl. Developeri môžu nastaviť metódu sledovania, ovplyvniť centrálne sledovacie parametre, definovať ktorá kamera sa používa na sledovanie na viac-kamerovom zariadení, definovať list modelov pri sledovaní viacerých modelov naraz, prepojiť rôzne vstupné zdroje, ktoré budú používané alebo nastaviť debug parametre počas vývoja aplikácie. Tieto konfiguračné súbory sú v podstate JSON súbory s presne danou štruktúrou.

Niektoré takéto konfiguračné parametre môžu slúžiť aj ako deklaratívne príkazy, ktoré môžeme napríklad použivať vnútri Unity GUI systéme na kontrolovanie alebo spúšťanie jednotlivých doplnkov a nakonfigurovať ich tam. Pre zhrnutie sú teda tieto konfiguračné súbory veľmi rýchlo deklarovateľným spôsbom ako delegovať a ovládať správanie VisionLib a ovplyvniť sledovancie výsledky bez komplikovaného písania kódu. 

Avšak ak chceme používať VisionLib je potrebné mať licenčný súbor, bez neho nie je možné používať VisionLib a taktiež ani testovať aplikáciu. Po registrácií je potebné tento súbor stiahnuť a vložiť do priečinku v Unity.

https://docs.visionlib.com/v3.0.1/

\subsection{S akým HW sa pracuje, porovnanie Android / iOS}

\section{Ciele práce}
\subsection{Stav riešenia problematiky na ÚAMT FEI STU}
\subsection{Vytýčenie cieľov diplomovej práce}

\section{Praktická časť diplomovej práce}
\subsection{Funkcionálne a nefunkcionálne požiadavky}
\subsection{Implementácia}